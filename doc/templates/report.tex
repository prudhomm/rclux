% -*- encoding: iso-latin-1 -*-

\documentclass{report}

\usepackage[latin1]{inputenc}
\usepackage[T1]{fontenc}
\usepackage{graphicx}
\usepackage[colorlinks]{hyperref}

\usepackage{makeidx}
\makeindex

%%%%%%%%%%%%%%%%%%%%%%%%%%%%%%
%% Glossaries
\usepackage[toc,section,acronym]{glossaries}
\newglossary{notation}{not}{ntn}{Notation}
\newglossaryentry{a}{name=$a$,description={Velocity of sound ($m/s$)},type=notation}

\newglossaryentry{singmtx}{name=Singular Matrix,
  description=A matrix with a zero determinant,
  first=singular matrix (SM),
  text=SM,
  firstplural=singular matrices (SMs)}
\newacronym{uv}{UV}{Ultra-Violet}
\newacronym{ujf}{UJF}{Universit� Joseph Fourier Grenoble 1}

\makeglossaries
%%%%%%%%%%%%%%%%%%%%%%%%%%%%%%

\title{Titre}

\author{Auteur}
\date{\today}


\begin{document}

\maketitle

\tableofcontents

\listoftables

\listoffigures

\chapter{Introduction}
\label{cha:introduction}

the keywords are \gls{uv}\index{Ultra-Violet} lamps\index{Ultra-Violet!lamp} irradiation\index{Ultra-Violet!irradiation}

use acronym \gls{ujf}

the speed of sound \gls{a}

\chapter{Mod�lisation}
\label{cha:modelisation}

\chapter{M�thodes num�riques}
\label{cha:methodes-numeriques}


\chapter{R�sultats num�riques}
\label{cha:resultats-numeriques}




\appendix

\chapter{Rappel de calcul vectoriel}
\label{cha:rappel-de-calcul}

blah blah

\chapter{Glossaries}
\label{cha:glossaries}


%% print the glossaries
\glossarystyle{list}
\printglossaries

%% print the index
\printindex

%% the next command allows to print _ALL_ references: you may not want
%% that in your report but rather have only the references you cited
\nocite{*}


\bibliographystyle{plain}
\bibliography{../biblio/rclux}




\end{document}


%%% Local Variables:
%%% coding: utf-8
%%% mode: latex
%%% TeX-PDF-mode: t
%%% TeX-parse-self: t
%%% x-symbol-8bits: nil
%%% TeX-auto-regexp-list: TeX-auto-full-regexp-list
%%% TeX-master: t
%%% ispell-local-dictionary: "french"
%%% End:
